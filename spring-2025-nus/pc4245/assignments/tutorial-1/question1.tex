\documentclass[main.tex]{subfiles}

\begin{document}

\begin{q}{1}
~
\begin{enumerate}[label=\textbf{(\alph*)}]
    \item From $c$, $\hbar$ and $G$ (Newton's constant of universal gravitation), construct a quantity $\ell_p$ with the dimension of length, a quantity $t_p$ with the dimension of time, a quantity $m_p$ with the dimension of mass. These are known as \textit{Planck length}, the  \textit{Planck time} and  \textit{Planck mass}, respectively, after Max Planck, who first published then in 1899 -- the year before the eponynmous constant itself. Work out the actual numbers in meters, seconds, and kilograms. Also calculate the  \textit{Planck energy} ($E_p = m_pc^2$) in \SI{}{\giga\eV}. [These quantities set the scale at which quantum gravity is expect to be relevant.]
    \item What is the gravitational analog to the fine structure constant? Find the actual number, using
    \begin{enumerate}[label=\roman*.]
        \item the mass of the electron,
        \item the Planck mass.
    \end{enumerate}
\end{enumerate}
[This question is from the D J Griffiths, Introduction to Elementary Particles, 2\textsuperscript{nd} Edition, Problem 12.9, page 420]
\end{q}

\begin{sol}

\end{sol}

\end{document}
