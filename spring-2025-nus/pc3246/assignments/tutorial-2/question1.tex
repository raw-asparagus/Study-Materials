\documentclass[main.tex]{subfiles}

\begin{document}

\begin{q}{1}
Explain how observing an inferior planet at greatest elongation can be used to
determine the planet's relative distance (i.e. the distance of the planet from
the Sun relative to Earth's distance from the Sun).
\end{q}

\begin{sol}
First, we define the average distance the Earth and the Inner Planet are from
the Sun as $1$ and $x$ astronomical unit(s) respectively where $x > 1$. Assuming
that the orbits of Earth and the Inner Planet around the Sun are perfectly
circular, we obtain the following geometric construction,
\begin{figure}[h!]
    \centering
    \includesvg[pretex=\fontsize{6}{6}\selectfont]{figure1}
\end{figure}

\noindent The Sun-Inner Planet-Earth triangle is a right triangle with the angle
subtended between the Sun-Inner Planet and Inner Planet-Earth lines being a
right angle. The angle $\theta$ subtended between the Inner Planet-Earth line
and the Sun-Earth line can be observationally obtained by measuring the angular
separation between the Inner Planet and the Sun as viewed from Earth. As such,
\begin{equation}
    \sin\theta = \frac{x}{1}\qquad \implies \qquad x = \sin\theta.
\end{equation}
\end{sol}

\end{document}
