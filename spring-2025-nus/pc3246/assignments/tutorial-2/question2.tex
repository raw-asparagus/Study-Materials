\documentclass[main.tex]{subfiles}

\begin{document}

\begin{q}{2}
Consider a superior planet. State the relationship among its synodic period,
sidereal period, and Earth's sidereal period. Explain the derivation of the
relationship.
\end{q}

\begin{sol}
Let the synodic period be $S$, the sidereal period be $T$, and the Earth's
sidereal period be $T_{\text{E}}$. Accordingly, we let the angular velocity of the superior period about the sun be $\omega$ and the angular velocity of the Earth about the Sun be $\omega_{\text{E}}$. As such, the synodic period can be considered to be the time $T$ needed for the angular displacement between the superior planet and Earth to be $2\pi$. Hence,
\begin{equation}
    2\pi = \omega S - \omega_{\text{E}}S\qquad \implies \qquad\frac{2\pi}{S} = \frac{2\pi}{T} - \frac{2\pi}{T_{\text{E}}}
\end{equation}
\begin{equation}
    \therefore\quad\frac{1}{S} = \frac{1}{T} - \frac{1}{T_{\text{E}}}.
\end{equation}
\end{sol}

\end{document}
