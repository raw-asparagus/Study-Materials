\documentclass[main.tex]{subfiles}

\begin{document}

\begin{q}{4}
If two lenses of focal lengths $f_1$ and $f_2$ can be considered to have zero physical separation, then the effective focal length fo the combination of lenses is
\begin{equation}
    \frac{1}{f_{\text{eff}}} = \frac{1}{f_1} + \frac{1}{f_2}.
\end{equation}
\begin{enumerate}[label=\textbf{(\alph*)}]
    \item Show that a compound lens system can be constructed from two lenses of different indices of refraction, $n_{1\lambda}$ and $n_{2\lambda}$, having the property that the resultant focal lengths of the compound lens at two specific wavelengths $\lambda_1$ and $\lambda_2$, respectively, can be made equal, or
    \begin{equation}
        f_{\text{eff},\lambda_1} = f_{\text{eff},\lambda_2}.
    \end{equation}
    \item Argue qualitatively that this condition does not guarantee that the focal length will be constant for all wavelengths.
\end{enumerate}
\end{q}

\begin{sol}
\subsubsection*{(a)}

\subsubsection*{(b)}
\end{sol}

\end{document}
