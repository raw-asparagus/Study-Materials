\documentclass[main.tex]{subfiles}

\begin{document}

\begin{q}{3}
Consider a planet with an axial tilt of \ang{90}. Does the Sun's gravitational
force on the equatorial bulge create a torque on the planet? Explain your
answer.
\end{q}

\begin{sol}
For a planet with an axial tilt of \ang{90}, the Sun's gravitational force on
the equatorial bulge does not create a torque on the planet.

\par\medskip

\noindent If a planet has an axial tilt of \ang{90} such that the Sun remains directly
overhead at one of its pole throughout its entire orbit, as the rotation axis of
the planet's rotation is parallel to the plane of the Solar System due to its
axial tilt of \ang{90}, the planet will observe equatorial bulges along a plane
perpendicular to the plane of the Solar System. Since the distance between any
point on the equatorial bulge and the Sun is the same, the gravitational force
experienced by the planet at any point of the equatorial bulge is symmetrically
the same as well. As such, there is no difference in gravitational force
experienced by any two opposite points on the equatorial bulge and hence the
Sun's gravitational force on the equatorial bulge does not create a torque on
the planet.
\end{sol}

\end{document}
