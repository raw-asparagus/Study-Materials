\documentclass[main.tex]{subfiles}

\begin{document}

\begin{q}{4}
Consider a planet with a retrograde rotation (i.e. clockwise rotation as seen
from above Earth's north pole). Does the Sun's gravitational torque cause the
planet's rotation axis to precess clockwise or counterclockwise as seen from
above Earth's north pole? Explain your answer.    
\end{q}

\begin{sol}
For a planet with retrograde rotation, the Sun's gravitational torque causes the
planet's rotation axis to precess counterclockwise as seen from above Earth's
north pole.

\par\medskip

\noindent For a planet with prograde rotation, the near side of the planet's
equatorial bulge from the Sun experiences a stronger gravitational force than
the far side. This results in a gravitational torque exerted onto the planet
about an axis that is perpendicular to the planet's axis of rotation and on the
plane of the Solar System. Since this torque does not change the rotation period
of the planet, the direction of the planet's axis of rotation changes
(precesses), resulting in a clockwise precession. Hence, if we consider a planet
with retrograde rotation, the angular momentum vector $\vb{L}$ describing the
rotation of the planet will now point in the opposite direction and thus the
planet will precess counterclockwise.

\par\medskip

\noindent Additionally as established in Q3, if the planet has an axial tilt of \ang{0} or
\ang{90}, there will be no precession.
\end{sol}

\end{document}
