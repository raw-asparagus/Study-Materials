\documentclass[main.tex]{subfiles}

\begin{document}

\begin{q}{1}
In general, an \textit{integral average} of some continuous function
$f\paren{t}$ over an interval $\tau$ is given by
\begin{equation}
    \aparen{f\paren{t}} = \frac{1}{\tau}\int_0^{\tau}f\paren{t}dt.
\end{equation}
Beginning with an expression for the integral average, prove that
\begin{equation}
    \aparen{U} = -G\frac{M\mu}{a},
\end{equation}
a binary system's gravitational potential energy, averaged over one period,
equals the value of the instantaneous potential energy of the system when the
two masses are separated by the distance $a$, the semimajor axis of the orbit of
the reduced mass about the center of mass.

\noindent\textit{Hints:
\begin{itemize}
    \item You may find the following definite integral useful:
    \begin{equation}
        \int_0^{2\pi}\frac{d\theta}{1 + e\cos\theta} = \frac{2\pi}{\sqrt{1 - e^2}}
    \end{equation}
    \item Express $U$ as a function of $\theta$, use $dt =
    \frac{dt}{d\theta}{d\theta}$ and $L = \mu r^2\frac{d\theta}{dt}$.
\end{itemize}}
\end{q}

\begin{sol}
Starting from the given integral average, we first perform a change of
variables,
\begin{equation}
    \aparen{U} = \frac{1}{\tau}\int_0^{\tau}U\paren{t}dt = \frac{1}{\tau}\int_0^{2\pi}U\paren{\theta}\frac{dt}{d\theta}d\theta.
\end{equation}
From the angular momenta $L$ of a reduced mass $\mu$ and central body $M$
subjected to a central force,
\begin{equation}
    L = \mu r^2\frac{d\theta}{dt}\qquad \implies \qquad\frac{dt}{d\theta} = \frac{\mu r^2}{L},
\end{equation}
and the reduced mass system will have a potential energy $U$,
\begin{equation}
    U\paren{r} = -\frac{GM\mu}{r}.
\end{equation}
Hence,
\begin{equation}
    \aparen{U} = \frac{1}{\tau}\int_0^{2\pi}\paren{-\frac{GM\mu}{r}}\frac{\mu r^2}{L}d\theta = -\frac{GM\mu^2}{L\tau}\int_0^{2\pi}rd\theta.
\end{equation}

\newpage
\noindent For a Keplerian orbit, the reduced mass $\mu$ draws out an ellipse about the central body $M$ as described by Kepler's 1\textsuperscript{st} Law,
\begin{equation}
    r\paren{\theta} = \frac{a\paren{1 - e^2}}{1 + e\cos\theta}.
\end{equation}
Therefore,
\begin{equation}
    \aparen{U} = -\frac{GM\mu^2}{L\tau}\int_0^{2\pi}\frac{a\paren{1 - e^2}}{1 + e\cos\theta}d\theta = -\frac{GM\mu^2a\paren{1 - e^2}}{L\tau}\frac{2\pi}{\sqrt{1 - e^2}} = -\frac{2\pi GM\mu^2a\sqrt{1 - e^2}}{L\tau},
\end{equation}
where by Kepler's 2\textsuperscript{nd} Law, an elliptical orbit has angular momenta $L$ given by,
\begin{equation}
    L = \mu\sqrt{GMa\paren{1 - e^2}}.
\end{equation}
Together,
\begin{equation}
    \aparen{U} = -\frac{2\pi GM\mu^2a\sqrt{1 - e^2}}{\mu\sqrt{GMa\paren{1 - e^2}}\tau} = -\frac{2\pi\mu\sqrt{GMa}}{\tau},
\end{equation}
and by Kepler's 3\textsuperscript{rd} Law, an elliptical orbit has period $\tau$ given by,
\begin{equation}
    \tau = 2\pi\sqrt{\frac{a^3}{GM}}.
\end{equation}
Such that,
\begin{equation}
    \aparen{U} = -\frac{2\pi\mu\sqrt{GMa}}{2\pi\sqrt{\frac{a^3}{GM}}} = -\frac{GM\mu}{a}.
\end{equation}

\end{sol}

\end{document}
