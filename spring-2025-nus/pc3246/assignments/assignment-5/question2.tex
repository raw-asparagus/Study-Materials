\documentclass[main.tex]{subfiles}

\begin{document}

\begin{q}{2}
Verify that Kepler's third law in the form of Eq. \ref{eq:keplers-3rd-law}
applies to the four moons that Galileo discovered orbiting Jupiter (the Galilean
moons: Io, Europa, Ganymede, and Callisto).
\begin{equation}\label{eq:keplers-3rd-law}
    P^2 = \frac{4\pi^2}{G\paren{m_1 + m_2}}a^3.
\end{equation}
\begin{enumerate}[label=\text{(\alph*)}]
    \item Using the data available in Appendix: Solar-System Data, create a
    graph of $\log_{10}P$ vs. $\log_{10}a$
    \item From the graph, show that the slope of the best-fit straight line
    through the data is $3/2$.
    \item Calculate the mass of Jupiter from the value of the $y$-intercept.
\end{enumerate}
\end{q}

\begin{sol}
\begin{subsol}
\begin{figure}[h!]
    \centering
    \includesvg[pretex=\fontsize{8}{8}\selectfont, width=\textwidth]{galilean_moons.svg}
\end{figure}
\end{subsol}

\newpage

\begin{subsol}
\begin{lstlisting}[style=pythonstyle]
# Linear regression
slope, intercept, r_value, p_value, std_err = linregress(galilean["log_10a"], galilean["log_10P"])

print("Linear regression results:")
print(f"\tSlope = {slope:.3f}")
print(f"\tIntercept = {intercept:.3f}")
\end{lstlisting}
\texttt{Linear regression results:}

\texttt{Slope = 1.49980}

\texttt{Intercept = -7.75139}

\begin{equation}
    m \approx 1.500\quad\paren{\text{4 s.f.}}
\end{equation}
\end{subsol}

\begin{subsol}
Since $m_{\text{J}} \gg m_{\text{moon}}$, we can simplify Kepler's third law as
\begin{equation}
    P^2 \approx \frac{4\pi^2}{Gm_{\text{J}}}a^3.
\end{equation}
Taking the $\log_{10}$ on both sides,
\begin{equation}
    \log_{10}P^2 = \log_{10}\paren{\frac{4\pi^2}{Gm_{\text{J}}}a^3}\qquad \implies \qquad2\log_{10}P = \log_{10}\paren{\frac{4\pi^2}{Gm_{\text{J}}}} + 3\log_{10}a.
\end{equation}
At the $y$-intercept of the $\log_{10}P$ vs. $\log_{10}a$ graph, $\log_{10}P =
-7.7514$ when $\log_{10}a = 3$. Hence,
\begin{equation}
    \log_{10}\paren{\frac{4\pi^2}{GM_{\text{J}}}} = 2\paren{-7.7514} - 3\paren{0}
\end{equation}
\begin{equation}
    \begin{split}
        M_{\text{J}} &= \frac{4\pi^2}{G\sparen{10^{2\paren{-7.7514}}}} \approx 1.88249 \times 10^{27}\SI{}{kg}\\
        &\approx 1.882 \times 10^{27}\SI{}{kg}\quad\paren{\text{4 s.f.}}
    \end{split}
\end{equation}
\end{subsol}
\end{sol}

\end{document}
