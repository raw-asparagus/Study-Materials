\documentclass[main.tex]{subfiles}

\begin{document}

\begin{q}{3}
Suppose that a radio telescope receiver has a bandwidth of \SI{50}{MHz} centered
at \SI{1.430}{GHz}. ($\SI{1}{GHz} = \SI{1000}{MHz}$). Assume that, rather than
being a perfect detector over the entire bandwidth, the receiver's frequency
dependence is triangular, meaning that the sensitivity of the detector is 0\% at
the edges of the band and 100\% at its center.
\begin{enumerate}[label=\text{(\alph*)}]
    \item Assume that the radio dish is a 100\% efficient reflector over the
    receiver's bandwidth and has a diameter of \SI{100}{m}. Assume also that the
    source NGC 2558 (a spiral galaxy with an apparent visual magnitude of
    $13.8$) has a constant spectral flux density of $S = \SI{2.5}{mJy}$ over the
    detector bandwidth. Calculate the total power measured at the receiver.
    \item Estimate the power emitted at the source in this frequency range if $d
    = \SI{100}{Mpc}$. Assume that the source emits the signal isotropically.
\end{enumerate}
\begin{notes}
$\SI{1}{Mpc} = 3.086 \times 10^{22}\SI{}{m}$
\end{notes}
\noindent\textit{Hints:
\begin{itemize}
    \item The integral of a triangular function is the area of a triangle.
    \item Does the power emitted at the source depend on $f_{\nu}$?
\end{itemize}}
\end{q}

\begin{sol}
\begin{subsol}
Since the receiver has a triangular frequency dependence, 
\begin{equation}
    \begin{split}
        P = S_0\sparen{\pi\paren{\frac{D}{2}}^2}\frac{\Delta f}{2} &= \paren{2.5 \times 10^{-29}}\sparen{\pi\paren{\frac{100}{2}}^2}\frac{50 \times 10^6}{2} \approx 4.909 \times 10^{-18}\SI{}{W}\\
        &\approx 4.9 \times 10^{-18}\SI{}{W}.
    \end{split}
\end{equation}
\end{subsol}

\begin{subsol}
Using the inverse square law,
\begin{equation}
    \begin{split}
        P_{\text{source}} &= \frac{P_{\text{receiver}}}{\pi\paren{\frac{D}{2}}^2}\paren{4\pi d^2} = \frac{16P_{\text{receiver}}d^2}{D^2} = \frac{16\paren{4.909 \times 10^{-18}}\paren{100 \times 3.0857 \times 10^{22}}^2}{100^2}\\
        &\approx 7.479 \times 10^{28}\SI{}{W}\\
        &\approx 7.5 \times 10^{28}\SI{}{W}.
    \end{split}
\end{equation}
\end{subsol}
\end{sol}

\end{document}
