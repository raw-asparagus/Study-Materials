\documentclass[main.tex]{subfiles}

\begin{document}

\begin{q}{4}
Consider an achromatic doublet. Let $R_1$ and $R_2$ be the radii of curvature of
the converging lens and $-R_2$ and $R_3$ be the radii the curvature of the
diverging lens.
\begin{enumerate}[label=\text{(\alph*)}]
    \item Express the focal length for red light ($f_{\text{r}}$) in terms of
    $R_1$, $R_2$, $R_3$, and the refractive indices of the crown and flint
    glasses.
    \item Express the focal length for blue light ($f_{\text{b}}$) in terms of
    $R_1$, $R_2$, $R_3$, and the refractive indices of the crown and flint
    glasses.
    \item Let $f_{\text{r}} = f_{\text{b}}$. Find an equation relating $R_1$,
    $R_2$, $R_3$. Simplify the equation.
\end{enumerate}
\end{q}

\begin{sol}
\begin{subsol}
For the crown (converging) lens made of crown glass with surfaces of radii $R_1$
and $R_2$, it has a focal length $f_{\text{c},~\text{r}}$ of,
\begin{equation}
    \frac{1}{f_{\text{c},~\text{r}}} = \paren{n_{\text{c},~\text{r}} - 1}\paren{\frac{1}{R_1} - \frac{1}{R_2}}.
\end{equation}
For the flint (diverging) lens with surfaces of radii $-R_2$ and $R_3$, it has a
focal length $f_{\text{f},~\text{r}}$ of,
\begin{equation}
    \frac{1}{f_{\text{f},~\text{r}}} = \paren{n_{\text{f},~\text{r}} - 1}\paren{\frac{1}{-R_2} - \frac{1}{R_3}} = \paren{n_{\text{f},~\text{r}} - 1}\paren{-\frac{1}{R_2} - \frac{1}{R_3}}.
\end{equation}
Thus, for red light
\begin{equation}
    \begin{split}
        \frac{1}{f_{\text{r}}} &= \frac{1}{f_{\text{c},~\text{r}}} + \frac{1}{f_{\text{f},~\text{r}}}\\
        &= \paren{n_{\text{c},~\text{r}} - 1}\paren{\frac{1}{R_1} - \frac{1}{R_2}} + \paren{n_{\text{f},~\text{r}} - 1}\paren{-\frac{1}{R_2} - \frac{1}{R_3}}.
    \end{split}
\end{equation}
\end{subsol}

\begin{subsol}
and for blue light,
\begin{equation}
    \frac{1}{f_{\text{b}}} = \paren{n_{\text{c},~\text{b}} - 1}\paren{\frac{1}{R_1} - \frac{1}{R_2}} + \paren{n_{\text{f},~\text{b}} - 1}\paren{-\frac{1}{R_2} - \frac{1}{R_3}}.
\end{equation}
\end{subsol}

\begin{subsol}
Let $f_{\text{r}} = f_{\text{b}}$,
\begin{equation}
    \begin{split}
        &\paren{n_{\text{c},~\text{r}} - 1}\paren{\frac{1}{R_1} - \frac{1}{R_2}} + \paren{n_{\text{f},~\text{r}} - 1}\paren{-\frac{1}{R_2} - \frac{1}{R_3}}\\
        &= \paren{n_{\text{c},~\text{b}} - 1}\paren{\frac{1}{R_1} - \frac{1}{R_2}} + \paren{n_{\text{f},~\text{b}} - 1}\paren{-\frac{1}{R_2} - \frac{1}{R_3}}
    \end{split}
\end{equation}
\begin{equation}
    \implies\quad\sparen{\paren{n_{\text{c},~\text{r}} - 1} - \paren{n_{\text{c},~\text{b}} - 1}}\paren{\frac{1}{R_1} - \frac{1}{R_2}} = \sparen{\paren{n_{\text{f},~\text{b}} - 1} - \paren{n_{\text{f},~\text{r}} - 1}}\paren{-\frac{1}{R_2} - \frac{1}{R_3}}
\end{equation}
\begin{equation}
    \implies\quad\paren{n_{\text{c},~\text{r}} - n_{\text{c},~\text{b}}}\paren{\frac{1}{R_1} - \frac{1}{R_2}} = \paren{n_{\text{f},~\text{b}} - n_{\text{f},~\text{r}}}\paren{-\frac{1}{R_2} - \frac{1}{R_3}}
\end{equation}
\begin{equation}
    \therefore\quad \paren{n_{\text{c},~\text{r}} - n_{\text{c},~\text{b}}}\paren{\frac{1}{R_1} - \frac{1}{R_2}} = \paren{n_{\text{f},~\text{r}} - n_{\text{f},~\text{b}}}\paren{\frac{1}{R_2} + \frac{1}{R_3}}.
\end{equation}
\end{subsol}
\end{sol}

\end{document}
