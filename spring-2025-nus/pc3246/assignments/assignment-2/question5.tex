\documentclass[main.tex]{subfiles}

\begin{document}

\begin{q}{5}
Consider an observer at the equator and a star at $A = \ang{70}$ and $h = \ang{0}$. Calculate the star's $A$ and $h$ after $1$ hour.

\noindent\textit{Hint: Assume that after $1$ hour, the star is at point $A$. Draw a great circle passing through the zenith and point $A$. Let $B$ be the point where the great circle intersects the horizon. Let the north celestial pole be point $C$. Consider the spherical triangle $ABC$. Assume that the radius of the celestial sphere is $1$ unit. Hence,the lengths of side $c$ and side $a$ correspond to the altitude and azimuth of the star, respectively.}
\begin{enumerate}[label=\text{(\alph*)}]
    \item What is the angle $B$?
    \item What is the angle $C$?
    \item What is the length of side $b$?
    \item By using a law of spherical trigonometry, calculate the length of side $c$.
    \item By using a law of spherical trigonometry, calculate the length of side $a$.
\end{enumerate}
\end{q}

\begin{sol}
\subsubsection*{(a)}

\subsubsection*{(b)}

\subsubsection*{(c)}

\subsubsection*{(d)}

\subsubsection*{(e)}    
\end{sol}

\end{document}
