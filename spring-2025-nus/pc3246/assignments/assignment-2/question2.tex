\documentclass[main.tex]{subfiles}

\begin{document}

\begin{q}{2}
Consider an observer at the latitude of \ang{40}N and a star at the azimuth $A =
\ang{0}$ and altitude $h = \ang{10}$. Find the star's $A$ and $h$ when it
transits the upper meridian.
\end{q}

\begin{sol}
Since the star has an azimuth of $A = \ang{0}$ with an altitude $h <
h_{\text{NCP}}$, the star is currently at lower culmination north of the prime
vertical with $h = \ang{10}$. Given a declination angle,
\begin{equation}
    \delta = h_{\text{lower culmination}} - \paren{\lambda - \ang{90}} = \ang{10} - \paren{\ang{40} - \ang{90}} = \ang{60}
\end{equation}
At upper culmination, the star will have an altitude of,
\begin{equation}
    h_{\text{upper culmination}} = \paren{\lambda + \ang{90}} - \delta = \paren{\ang{40} + \ang{90}} - \ang{60} = \ang{70}
\end{equation}
and an azimuth $A_{\text{upper culmination}} = \ang{0}$ since the star will
remain north of the prime vertical.
\end{sol}

\end{document}
