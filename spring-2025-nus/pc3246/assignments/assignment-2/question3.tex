\documentclass[main.tex]{subfiles}

\begin{document}

\begin{q}{3}
Consider an observer at the equator and a star at $A = \ang{90}$ and $h =
\ang{0}$ at 8 pm after $4$ months.
\end{q}

\begin{sol}
Since the observer is at the equator, the point $\paren{A = \ang{0},~h =
\ang{0}}$ coincides with the north celestial pole, for a star viewed by an
observer at the equator lying due east,
\begin{equation}
    \delta = \ang{0}\qquad \text{and} \qquad\text{HA} = \ra{-6}.
\end{equation}
As such, after $4$ months at 8 pm,
\begin{equation}
    \delta^{\prime} = \delta = \ang{0}\qquad \text{and} \qquad\text{HA}^{\prime} = \ra{-6} + \ra{8} = \ra{2},
\end{equation}
and thus a local observer will observe the star at.
\begin{equation}
    A = \ang{270}\qquad \text{and} \qquad h = \ang{60}.
\end{equation}
\end{sol}

\end{document}
