\documentclass[main.tex]{subfiles}

\begin{document}

\begin{q}{1}
Consider the planet Mars. Calculate its relative distance by using the following observational data. Do not use Kepler's laws of plaentary motion and Newton's law of universal gravitation (assume that they have yet to be discovered).

\noindent\textit{Hint: Consider the reference frame in which the Sun and Earth are not moving. Find the angle subtended at the Sun by two of the phenomena. Use trigonometry to calculate Mars' relative distance.}

\begin{table}[h!]
    \centering
    \begin{tabular}{|c|c|}
    \hline
    \textbf{Date} & \textbf{Phenomenon} \\
    \hline
    2023-11-18 & Conjunction \\
    2024-10-14 & Western quadrature \\
    2025-01-16 & Opposition \\
    2025-04-21 & Eastern quadrature \\
    \hline
    \end{tabular}
    \caption*{Reference: \href{https://eco.mtk.nao.ac.jp/cgi-bin/koyomi/cande/phenomena_en.cgi}{https://eco.mtk.nao.ac.jp/cgi-bin/koyomi/cande/phenomena\_en.cgi}}
    \end{table}
\end{q}

\begin{sol}

\end{sol}

\end{document}
