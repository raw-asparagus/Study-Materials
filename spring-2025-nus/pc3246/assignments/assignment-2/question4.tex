\documentclass[main.tex]{subfiles}

\begin{document}

\begin{q}{4}
Consider an observer at the latitude of \ang{20}N and a star at $A = \ang{180}$
and $h = \ang{60}$. If the vernal equinox is at its meridian two hours later,
what are the declination and right ascension of the star?
\end{q}

\begin{sol}
Since the star has an azimuth of $A = \ang{180}$, the star is currently at upper
culmination south of the prime vertical with an altitude $h = \ang{60}$ and a
declination angle,
\begin{equation}
    \delta = \paren{\lambda + \ang{90}} - \paren{\ang{180} - h_{\text{upper culmination}}} = \paren{\ang{20} + \ang{90}} - \paren{\ang{180} - \ang{60}} = \ang{-10}.
\end{equation}
Consider the local sidereal time (LST),
\begin{equation}
    \text{LST} = \text{HA}_{\aries} = \ra{-2},
\end{equation}
and the hour angle of the star,
\begin{equation}
    \text{HA} = \ra{0},
\end{equation}
which gives,
\begin{equation}
    \alpha = \text{LST} - \text{HA} = \ra{-2} - \ra{0} = \ra{-2}.
\end{equation}
\end{sol}

\end{document}
