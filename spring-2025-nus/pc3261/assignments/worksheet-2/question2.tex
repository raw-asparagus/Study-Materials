\documentclass[main.tex]{subfiles}

\begin{document}

\begin{ex}{1.3}
Find the tangent, normal and binormal vectors, as well as, curvature and torsion
for the circular helix.
\end{ex}

\begin{sol}
Starting with the position vector of a moving particle with a trajectory of a
circular helix,
\begin{equation}
    \vb{r} = a\cos\omega t\hat{\vb{e}}_x + a\sin\omega t\hat{\vb{e}}_y + b\omega t\hat{\vb{e}}_z.
\end{equation}
From the definition of the velocity vector as the rate of change of the position
vector w.r.t. time,
\begin{equation}
    \vb{v} = \frac{d\vb{r}}{dt} = -a\omega\sin\omega t\hat{\vb{e}}_x + a\omega\cos\omega t\hat{\vb{e}}_y + b\omega\hat{\vb{e}}_z.
\end{equation}
From which, we can obtain the trajectory arc length w.r.t. $t = 0$,
\begin{equation}
    s = \int_0^t\abs{\vb{v}}dt^{\prime} = \int_0^t\abs{-a\omega\sin\omega t^{\prime} + a\omega\cos\omega t^{\prime} + b\omega}dt^{\prime} = \omega t\sqrt{a^2 + b^2},
\end{equation}
and,
\begin{equation}
    \frac{ds}{dt} = \omega\sqrt{a^2 + b^2}.
\end{equation}
\newpage\noindent
Given the definition of the tangent vector,
\begin{equation}
    \begin{split}
        \hat{\vb{e}}_{\text{T}} &= \frac{d\vb{r}}{ds} = \frac{d\vb{r}}{dt}\frac{dt}{ds} = \paren{-a\omega\sin\omega t\hat{\vb{e}}_x + a\omega\cos\omega t\hat{\vb{e}}_y + b\omega\hat{\vb{e}}_z}\paren{\frac{1}{\omega\sqrt{a^2 + b^2}}}\\
        &= \frac{1}{\sqrt{a^2 + b^2}}\paren{-a\sin\omega t\hat{\vb{e}}_x + a\cos\omega t\hat{\vb{e}}_y + b\hat{\vb{e}}_z},
    \end{split}
\end{equation}
and,
\begin{equation}
    \frac{d\hat{\vb{e}}_{\text{T}}}{dt} = \frac{1}{\sqrt{a^2 + b^2}}\paren{-a\omega\cos\omega t\hat{\vb{e}}_x - a\omega\sin\omega t\hat{\vb{e}}_y} = -\frac{a\omega}{\sqrt{a^2 + b^2}}\paren{\cos\omega t\hat{\vb{e}}_x + \sin\omega t\hat{\vb{e}}_y}
\end{equation}
\begin{equation}
    \begin{split}
        \implies\quad \frac{d\hat{\vb{e}}_{\text{T}}}{ds} = \frac{d\hat{\vb{e}}_{\text{T}}}{dt}\frac{dt}{ds} &= -\frac{a\omega}{\sqrt{a^2 + b^2}}\paren{\cos\omega t\hat{\vb{e}}_x + \sin\omega t\hat{\vb{e}}_y}\paren{\frac{1}{\omega\sqrt{a^2 + b^2}}}\\
        &= -\frac{a}{a^2 + b^2}\paren{\cos\omega t\hat{\vb{e}}_x + \sin\omega t\hat{\vb{e}}_y}.
    \end{split}
\end{equation}
Given the definition of the normal vector,
\begin{equation}
    \hat{\vb{e}}_{\text{N}} \equiv \underbrace{\abs{\frac{1}{\frac{d\hat{\vb{e}}_{\text{T}}}{ds}}}}_{1 / \kappa}\frac{d\hat{\vb{e}}_{\text{T}}}{ds}.
\end{equation}
\begin{equation}
    \begin{split}
        \therefore\quad \kappa = \abs{\frac{d\hat{\vb{e}}_{\text{T}}}{ds}} &= \frac{a}{a^2 + b^2}\paren{\cos^2\omega t + \sin^2\omega t}\\
        &= \frac{a}{a^2 + b^2}.
    \end{split}
\end{equation}
Hence,
\begin{equation}
    \begin{split}
        \hat{\vb{e}}_{\text{N}} = \frac{1}{\kappa}\frac{d\hat{\vb{e}}_{\text{T}}}{ds} &= \frac{a^2 + b^2}{a}\sparen{-\frac{a}{a^2 + b^2}\paren{\cos\omega t\hat{\vb{e}}_x + \sin\omega t\hat{\vb{e}}_y}}\\
        &= -\paren{\cos\omega t\hat{\vb{e}}_x + \sin\omega t\hat{\vb{e}}_y},
    \end{split}
\end{equation}
and,
\begin{equation}
    \begin{split}
        \frac{d\hat{\vb{e}}_{\text{N}}}{ds} = \frac{d\hat{\vb{e}}_{\text{N}}}{dt}\frac{dt}{ds} &= -\paren{-\omega\sin\omega t\hat{\vb{e}}_x + \omega\cos\omega t\hat{\vb{e}}_y}\paren{\frac{1}{\omega\sqrt{a^2 + b^2}}}\\
        &= \frac{1}{\sqrt{a^2 + b^2}}\paren{\sin\omega t\hat{\vb{e}}_x - \cos\omega t\hat{\vb{e}}_y}.
    \end{split}
\end{equation}
\newpage\noindent
Given the definition of the binormal vector as orthonormal to the tangent and
normal vectors,
\begin{equation}
    \begin{split}
        \hat{\vb{e}}_{\text{B}} \equiv \hat{\vb{e}}_{\text{T}} \times \hat{\vb{e}}_{\text{N}} &= \sparen{\frac{1}{\sqrt{a^2 + b^2}}\paren{-a\sin\omega t\hat{\vb{e}}_x + a\cos\omega t\hat{\vb{e}}_y} + b\hat{\vb{e}}_z} \times \sparen{-\paren{\cos\omega t\hat{\vb{e}}_x + \sin\omega t\hat{\vb{e}}_y}}\\
        &= -\frac{1}{\sqrt{a^2 + b^2}}\begin{vmatrix}
            \hat{\vb{e}}_x & \hat{\vb{e}}_y & \hat{\vb{e}}_z \\
            -a\sin\omega t & a\cos\omega t & b \\
            \cos\omega t & \sin\omega t & 0
        \end{vmatrix}\\
        &= -\frac{1}{\sqrt{a^2 + b^2}}\sparen{-b\sin\omega t\hat{\vb{e}}_x - \paren{-b\cos\omega t\hat{\vb{e}}_y} + \paren{-a\sin^2\omega t - a\cos^2\omega t}\hat{\vb{e}}_z}\\
        &= \frac{1}{\sqrt{a^2 + b^2}}\paren{b\sin\omega t\hat{\vb{e}}_x - b\cos\omega t\hat{\vb{e}}_y + a\hat{\vb{e}}_z},
    \end{split}
\end{equation}
and,
\begin{equation}
    \frac{d\hat{\vb{e}}_{\text{B}}}{ds} \equiv -\tau\hat{\vb{e}}_{\text{N}}.
\end{equation}
Since the set of basis vectors
$\cparen{\hat{\vb{e}}_{\text{T}},~\hat{\vb{e}}_{\text{N}},~\hat{\vb{e}}_{\text{B}}}$
are mutually orthonormal,
\begin{equation}
    \hat{\vb{e}}_{\text{N}} \cdot \hat{\vb{e}}_{\text{B}} = 0,
\end{equation}
and thus,
\begin{equation}
    \hat{\vb{e}}_{\text{N}} \cdot \underbrace{\frac{d\hat{\vb{e}}_{\text{B}}}{ds}}_{-\tau\hat{\vb{e}}_{\text{N}}} + \frac{d\hat{\vb{e}}_{\text{N}}}{ds} \cdot \hat{\vb{e}}_{\text{B}} = 0
\end{equation}
\begin{equation}
    \begin{split}
        \implies \quad \tau &= -\frac{d\hat{\vb{e}}_{\text{N}}}{ds} \cdot \hat{\vb{e}}_{\text{B}}\\
        &= -\sparen{-\frac{1}{\sqrt{a^2 + b^2}}\paren{\sin\omega t\hat{\vb{e}}_x - \cos\omega t\hat{\vb{e}}_y}} \cdot \sparen{\frac{1}{\sqrt{a^2 + b^2}}\paren{b\sin\omega t\hat{\vb{e}}_x - b\cos\omega t\hat{\vb{e}}_y + a\hat{\vb{e}}_z}}\\
        &= \frac{1}{a^2 + b^2}\paren{b\sin^2\omega t + b\cos^2\omega t}\\
        &= \frac{b}{a^2 + b^2}.
    \end{split}
\end{equation}
\end{sol}

\end{document}
