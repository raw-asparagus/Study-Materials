\documentclass[main.tex]{subfiles}

\begin{document}

\begin{ex}{1.4}
Establish the relationship between unit basis vectors
$\paren{\hat{\vb{e}}_{\rho},~\hat{\vb{e}}_{\phi}}$ of the polar coordinate
system and the unit basis vectors $\paren{\hat{\vb{e}}_x,~\hat{\vb{e}}_y}$ of
the Cartesian coordinate system. 
\end{ex}

\begin{sol}
Geometrically,
\begin{equation}
    \begin{dcases}
        ~\hat{\vb{e}}_{\rho} = \cos\phi\hat{\vb{e}}_x + \sin\phi\hat{\vb{e}}_y \\
        ~\hat{\vb{e}}_{\phi} = -\sin\phi\hat{\vb{e}}_x + \cos\phi\hat{\vb{e}}_y
    \end{dcases}.
\end{equation}
This transformation can be cast into a transformation matrix $\bm{R}$ as,
\begin{equation}
    \bm{R} = \begin{pmatrix}
        \cos\phi & \sin\phi \\
        -\sin\phi & \cos\phi
    \end{pmatrix}.
\end{equation}
\newpage\noindent
Since this transformation matrix is a rotation matrix,
\begin{equation}
    \bm{R}^{-1} = \bm{R}^{\intercal} = \begin{pmatrix}
        \cos\phi & -\sin\phi \\
        \sin\phi & \cos\phi
    \end{pmatrix}
\end{equation}
\begin{equation}
    \implies\quad \begin{pmatrix}
        \hat{\vb{e}}_x \\
        \hat{\vb{e}}_y
    \end{pmatrix} = \bm{R}^{-1}\begin{pmatrix}
        \hat{\vb{e}}_{\rho} \\
        \hat{\vb{e}}_{\phi}
    \end{pmatrix} = \begin{pmatrix}
        \cos\phi & -\sin\phi \\
        \sin\phi & \cos\phi
    \end{pmatrix}\begin{pmatrix}
        \hat{\vb{e}}_{\rho} \\
        \hat{\vb{e}}_{\phi}
    \end{pmatrix} = \begin{pmatrix}
        \cos\phi\hat{\vb{e}}_{\rho} - \sin\phi\hat{\vb{e}}_{\phi} \\
        \sin\phi\hat{\vb{e}}_{\rho} + \cos\phi\hat{\vb{e}}_{\phi}
    \end{pmatrix},
\end{equation}
and,
\begin{equation}
    \begin{dcases}
        ~\hat{\vb{e}}_x = \cos\phi\hat{\vb{e}}_{\rho} - \sin\phi\hat{\vb{e}}_{\phi} \\
        ~\hat{\vb{e}}_y = \sin\phi\hat{\vb{e}}_{\rho} + \cos\phi\hat{\vb{e}}_{\phi}
    \end{dcases}.
\end{equation}
\end{sol}

\end{document}
