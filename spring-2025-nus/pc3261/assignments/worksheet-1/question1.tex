\documentclass[main.tex]{subfiles}

\begin{document}

\begin{ex}{1.1}
Find the particles's velocity and acceleration vectors. What are the magnitude
and direction of the particle's acceleration?
\end{ex}

\begin{sol}
Starting with the position vector of the particle undergoing uniform circular
motion,
\begin{equation}
    \vb{r} = R\cos\omega t\hat{\vb{e}}_x + R\sin\omega t\hat{\vb{e}}_y.
\end{equation}
From the definition of the velocity vector as the rate of change of the position
vector w.r.t. time,
\begin{equation}
    \begin{split}
        \vb{v} \equiv \frac{d\vb{r}}{dt} &= R\paren{-\omega\sin\omega t\hat{\vb{e}}_x + \omega\cos\omega t\hat{\vb{e}}_y}\\
        &= R\omega\paren{-\sin\omega t\hat{\vb{e}}_x + \cos\omega t\hat{\vb{e}}_y}.
    \end{split}
\end{equation}
Similarly, from the definition of the acceleration vector as the rate of change
of the velocity vector w.r.t. time,
\begin{equation}
    \begin{split}
        \vb{a} \equiv \frac{d\vb{v}}{dt} &= R\omega\paren{-\omega\cos\omega t\hat{\vb{e}}_x - \omega\sin\omega t\hat{\vb{e}}_y}\\
        &= -R\omega^2\paren{\cos\omega t\hat{\vb{e}}_x + \sin\omega t\hat{\vb{e}}_y},
    \end{split}
\end{equation}
which yields a magnitude,
\begin{equation}
    \begin{split}
        a = \sqrt{\vb{a} \cdot \vb{a}} &= \sqrt{R^2\omega^4\cos^2\omega t\underbrace{\paren{\hat{\vb{e}}_x \cdot \hat{\vb{e}}_x}}_1 + R^2\omega^4\sin^2\omega t\underbrace{\paren{\hat{\vb{e}}_y \cdot \hat{\vb{e}}_y}}_1}\\
        &= R\omega^2,
    \end{split}
\end{equation}
and a direction,
\begin{equation}
    \hat{\vb{a}} = \frac{\vb{a}}{a} = -\cos\omega t\hat{\vb{e}}_x - \sin\omega t\hat{\vb{e}}_y,
\end{equation}
pointing radially inwards towards the centre of the circular path.
\end{sol}

\end{document}
